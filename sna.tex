
\documentclass[11pt,twoside, book]{book}
\usepackage[mono=false]{libertine} % new linux font, ignore mono

\usepackage{luatex85}

%\renewcommand{\baselinestretch}{1.05}
\usepackage{amsmath,amsthm,amssymb,mathrsfs,amsfonts,dsfont}
\usepackage{epsfig,graphicx}
\usepackage{tabularx}
\usepackage{blkarray}
\usepackage{slashed}
\usepackage{hyperref}
\usepackage{color}
\usepackage{listings}
\usepackage{caption}
\usepackage[toc,title,titletoc,header]{appendix}
\usepackage{minitoc}
\usepackage{color}
\usepackage{multicol} % two-col ToC
\usepackage{bm}
\usepackage{fixltx2e,fix-cm}
\usepackage{graphicx}
\usepackage{subfigure}
\usepackage{makeidx}


% link colors settings
\hypersetup{
    colorlinks=true,
    citecolor=magenta,
    linkcolor=blue,
    filecolor=green,      
    urlcolor=cyan,
    % hypertexnames=false,
}
\usepackage[capitalise]{cleveref}
\usepackage{subcaption}
\usepackage{enumitem}
\usepackage{mathtools}
\usepackage{physics}
\usepackage[linesnumbered,ruled,vlined,algosection]{algorithm2e}
\SetCommentSty{textsf}
\usepackage{epigraph}
\epigraphwidth=1.0\linewidth
\epigraphrule=0pt

% adjust margin
\usepackage[margin=2.3cm]{geometry}
\headheight13.6pt

%%%%%%%%%%%%%%%% thmtools %%%%%%%%%%%%%%%%%%%%%
\usepackage{thmtools}
\declaretheorem[numberwithin=chapter]{theorem}
\declaretheorem[numberwithin=chapter]{axiom}
\declaretheorem[numberwithin=chapter]{lemma}
\declaretheorem[numberwithin=chapter]{proposition}
\declaretheorem[numberwithin=chapter]{claim}
\declaretheorem[numberwithin=chapter]{conjecture}
\declaretheorem[sibling=theorem]{corollary}
\declaretheorem[numberwithin=chapter, style=definition]{definition}
\declaretheorem[numberwithin=chapter, style=definition]{problem}
\declaretheorem[numberwithin=chapter, style=definition]{example}
\declaretheorem[numberwithin=chapter, style=definition]{exercise}
\declaretheorem[numberwithin=chapter, style=definition]{observation}
\declaretheorem[numberwithin=chapter, style=definition]{fact}
\declaretheorem[numberwithin=chapter, style=definition]{construction}
\declaretheorem[numberwithin=chapter, style=definition]{remark}
\declaretheorem[numberwithin=chapter, style=remark]{question}
%%%%%%%%%%%%%%%% thmtools %%%%%%%%%%%%%%%%%%%%%
\usepackage{changepage}
\newenvironment{solution}
    {\renewcommand\qedsymbol{$\square$}\color{blue}\begin{adjustwidth}{0em}{2em}\begin{proof}[\textit Solution.~]}
    {\end{proof}\end{adjustwidth}}

%%%%%%%%%%%%%%%% index %%%%%%%%%%%%%%%%%%%%%
\begin{filecontents}{index.ist}
% https://tex.stackexchange.com/questions/65247/index-with-an-initial-letter-of-the-group
headings_flag 1
heading_prefix "{\\centering\\large \\textbf{"
heading_suffix "}}\\nopagebreak\n"
delim_0 "\\nobreak\\dotfill"
\end{filecontents}
\newcommand{\myindex}[1]{\index{#1} \emph{#1}}
\makeindex[columns=3, intoc, title=Alphabetical Index, options= -s index.ist]
%%%%%%%%%%%%%%%% index %%%%%%%%%%%%%%%%%%%%%

%%%%%%%%%%%%%%%% ToC %%%%%%%%%%%%%%%%%%%%%
% Link Chapter title to ToC: https://tex.stackexchange.com/questions/32495/linking-the-section-text-to-the-toc
\usepackage[explicit]{titlesec}
\titleformat{\chapter}[display]
  {\normalfont\huge\bfseries}{\chaptertitlename\ {\thechapter}}{20pt}{\hyperlink{chap-\thechapter}{\Huge#1}
\addtocontents{toc}{\protect\hypertarget{chap-\thechapter}{}}}
\titleformat{name=\chapter,numberless}
  {\normalfont\huge\bfseries}{}{-20pt}{\Huge#1}

%%%%%%%%%%%%%%%%%%% fancyhdr %%%%%%%%%%%%%%%%%
\usepackage{fancyhdr}
\pagestyle{fancy} % enable fancy page style
\renewcommand{\headrulewidth}{0.0pt} % comment if you want the rule
\fancyhf{} % clear header and footer
\fancyhead[lo,le]{\leftmark}
\fancyhead[re,ro]{\rightmark}
\fancyfoot[CE,CO]{\hyperref[toc-contents]{\thepage}}

% https://tex.stackexchange.com/questions/550520/making-each-page-number-link-back-to-beginning-of-chapter-or-section
\makeatletter
\def\chaptermark#1{\markboth{\protect\hyper@linkstart{link}{\@currentHref}{Chapter \thechapter ~ #1}\protect\hyper@linkend}{}}
\def\sectionmark#1{\markright{\protect\hyper@linkstart{link}{\@currentHref}{\thesection ~ #1}\protect\hyper@linkend}}
\makeatother
%%%%%%%%%%%%%%%%%%% fancyhdr %%%%%%%%%%%%%%%%%


%%%%%%%%%%%%%%%%%%% biblatex %%%%%%%%%%%%%%%%%
\usepackage[doi=false,url=false,isbn=false,style=alphabetic,backend=biber,backref=true]{biblatex}
\addbibresource{bib.bib}

\newbibmacro{string+doiurlisbn}[1]{%
  \iffieldundef{doi}{%
    \iffieldundef{url}{%
      \iffieldundef{isbn}{%
        \iffieldundef{issn}{%
          #1%
        }{%
          \href{http://books.google.com/books?vid=ISSN\thefield{issn}}{#1}%
        }%
      }{%
        \href{http://books.google.com/books?vid=ISBN\thefield{isbn}}{#1}%
      }%
    }{%
      \href{\thefield{url}}{#1}%
    }%
  }{%
    \href{http://dx.doi.org/\thefield{doi}}{#1}%
  }%
}

% https://tex.stackexchange.com/questions/94089/remove-quotes-from-inbook-reference-title-with-biblatex
\DeclareFieldFormat[article,incollection,inproceedings,book,misc]{title}{\usebibmacro{string+doiurlisbn}{\mkbibemph{#1}}}
% https://tex.stackexchange.com/questions/454672/biblatex-journal-name-non-italic
\DeclareFieldFormat{journaltitle}{#1\isdot}
\DeclareFieldFormat{booktitle}{#1\isdot}
\renewbibmacro{in:}{}
% add video field: https://tex.stackexchange.com/questions/111846/biblatex-2-custom-fields-only-one-is-working
\DeclareSourcemap{
    \maps[datatype=bibtex]{
      \map{
        \step[fieldsource=video]
        \step[fieldset=usera,origfieldval]
    }
  }
}
\DeclareFieldFormat{usera}{\href{#1}{\textsc{Online video}}}
\AtEveryBibitem{
    \csappto{blx@bbx@\thefield{entrytype}}{% put at end of entry
        \iffieldundef{usera}{}{\space \printfield{usera}}
    }
}
%%%%%%%%%%%%%%%%%%% biblatex %%%%%%%%%%%%%%%%%

%%%%%%%%%%%%%%%%%%%%% glossaries %%%%%%%%%%%%%%%%%
\input{./glossaries.tex}
%%%%%%%%%%%%%%%%%%%%% glossaries %%%%%%%%%%%%%%%%%

%%%%%%%%%%%%%%%%%%%%% glossaries-extra %%%%%%%%%%%%%%%%%
% \usepackage[record,abbreviations,symbols,stylemods={list,tree,mcols}]{glossaries-extra}
%%%%%%%%%%%%%%%%%%%%% glossaries-extra %%%%%%%%%%%%%%%%%


\input{./macros.tex}

%%%%%%%%%%%%%%%%%%%%%%%%%%%%%%%%%%%%%%%%%%%%%%%%%%
%%%%%%%%%%%%%%%% begin of document %%%%%%%%%%%%%%%
%%%%%%%%%%%%%%%%%%%%%%%%%%%%%%%%%%%%%%%%%%%%%%%%%%

\begin{document}

\title{\bf \huge Social Network analysis}
\author{Scott Turner}
\date{Update on \today}
\maketitle
\setcounter{tocdepth}{2}
\setcounter{minitocdepth}{1} 

\begin{multicols}{2}
    \dominitoc% Initialization
    \adjustmtc[2]% chp number shift for mini-toc
    \tableofcontents
    \label{toc-contents}
\end{multicols}

	\listoffigures
	% \listoftables
\begin{multicols}{2}
	\listoftheorems[ignoreall,show={theorem}]
\end{multicols}

	\renewcommand{\listtheoremname}{List of Definitions}
\begin{multicols}{2}
	\listoftheorems[ignoreall,show={definition}]
\end{multicols}

\input{chapter/Introduction/introduction}

\part{Useful other tools}
\chapter{Gephi}
\section{Introduction}
Gephi is not a social Media analysis tool it is a tool for analysing and visualising networks.

\chapter{Python}
\section{Sentiment analysis}
The file socmedhe20.csv contains the tweets and the code is a  a modified example taken from Wintjen M (2020) "Practical Analysis Using Jupyter Notebook" pp 264

\begin{verbatim}
! pip install nltk
import nltk
import pandas as pd
import numpy as np
%matplotlib inlinefrom nltk.sentiment.vader import SentimentIntensityAnalyzer
anlysr=SentimentIntensityAnalyzer()
nltk.download('vader_lexicon')
from nltk.sentiment.vader import SentimentIntensityAnalyzer
anlysr=SentimentIntensityAnalyzer()
the_data=pd.read_csv('socmedhe20.csv')
the_data.head()
score_compound=[]
score_positive=[]
score_negative=[]
score_neutral=[]
i=0
while (i<len(the_data)):
    my_anlysr=anlysr.polarity_scores(the_data.iloc[i]['text'])
    score_compound.append(my_anlysr['compound'])
    score_positive.append(my_anlysr['pos'])
    score_negative.append(my_anlysr['neg'])
    score_neutral.append(my_anlysr['neu'])
    i=i+1
the_data['Compound score']=score_compound
the_data['Positive score']=score_positive
the_data['Negative score']=score_negative
the_data['Neutral score']=score_neutral
loop=0
pred_sentiment=[]
while (loop<len(the_data)):
    if ((the_data.iloc[loop]['Compound score'])>0.3):
        pred_sentiment.append('Positive Words')
    elif ((the_data.iloc[loop]['Compound score']>=0) & (the_data.iloc[loop]['Compound score']<0.3)):
        pred_sentiment.append('Neutral Words')
    else:
        pred_sentiment.append('Negative Words')
    loop=loop+1
the_data['Prediction']=pred_sentiment
the_data.groupby('Prediction').size().plot(kind='barh')    
\end{verbatim}
 





\part{TAGS}
\chapter{TAGS}
\section{Introduction}
The main tool is TAGS developed by Martin Hawksey (Hawksey, 2022). TAGS is a way of capturing tweets for a particular hashtag, initially upto the previous 7 days, and capturing them as an archive in Google Sheets. One of the great features is once you set it up, you can leave it to collect data every hour and add to the archive. This approach was used in all the data sets used in this paper.

\section{Setting up a TAGS sheet}
Setting up a TAGS sheet is relatively easy go to website (Hawksey, 2022), follow the Get TAGS link and choose which version you want to use. It is worth reading and following the guidance on the page about the app not verified. You will need an Twitter account to link to. 
Figure 1 shows the first page of the TAGS sheet after the hashtags you want to consider (box 2) have been eunning for a while.

So in summary the steps are
•	Go to https://tags.hawksey.info/ create a new spreadsheet via Get TAGS and select TAGS v6.1
•	Make a copy
•	In box 2 enter the search term e.g. #socmedhe21
•	Use the TAGS pull down menu select run now! 
•	Whole load of authorization
•	In TAGS menu update hourly
•	Share button: so all can view it. \url{https://docs.google.com/spreadsheets/d/1o49hgPpud99FdVD-DxShD1B_v4fuPdTZoTdrVZFCjXI/edit?usp=sharing} 


\part{Socioviz}
\input{chapter/Chpsocioviz1/sociovizintro}
\chapter{From socioviz to Gephi}
Starting with a Socioviz report instead of (or after viewing the report) viewing the report you can download it. A zip file of CSVs and gexf files is produced. \paragraph{}

\section{Dealing with gexf files}Going to concentrate on the gexf ones\paragraph{}

An example is shown in figure 1.1
\begin{figure}
    \centering
    \includegraphics[width=10cm]{chapter/Chpsocioviz1/gephi1.png}
    \caption{Selecting a gexf file}
    \label{fig:GexfSelection}
\end{figure}
\newline
Gexf files can be load straight into GEPHI without any real difficulty. Put when it is loaded it may not be in the format you want (see figure 1.2). \paragraph{}
\begin{figure}
    \centering 
    \includegraphics[width=10cm]{chapter/Chpsocioviz1/gephi2.png}
    \caption{After loading}
    \label{fig:afterload}
\end{figure}
It is difficult to see what is going on all the nodes are on top of each other. So now we play with the layout to get them spread out (see figure 1.3).\paragraph{}
\begin{figure}
    \centering
    \includegraphics[width=10cm]{chapter/Chpsocioviz1/gephi3.png}
    \caption{Change the Layout}
    \label{fig:layoutchange}
\end{figure}
Ok, but what do the points mean though. If you click on the T icon near the bottom of the screen it adds the names of the nodes (see figure 1.4).\paragraph{}
\begin{figure}
    \centering
    \includegraphics[width=10cm]{chapter/Chpsocioviz1/gephi5.png}
    \caption{Add the labels}
    \label{fig:addlabels}
\end{figure}
It is difficult to see so of the nodes so one trick is to make them all the same size using the Appear, node and Unique here they are all set to 10 (see figure 1.5).\paragraph{}
\begin{figure}
    \centering
    \includegraphics[width=10cm]{chapter/Chpsocioviz1/gephi6.png}
    \caption{Resize the nodes}
    \label{fig:nodes resize}
\end{figure}


\part{NodeXL}
\input{chapter/ChpNodeXL/NodeXL}





\backmatter

%%%%%%%%%%%%%%% Reference %%%%%%%%%%%%%%%

\printbibliography[heading=bibintoc]
\printindex

\end{document}

