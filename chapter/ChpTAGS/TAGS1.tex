\chapter{TAGS}
\section{Introduction}
The main tool is TAGS developed by Martin Hawksey (Hawksey, 2022). TAGS is a way of capturing tweets for a particular hashtag, initially upto the previous 7 days, and capturing them as an archive in Google Sheets. One of the great features is once you set it up, you can leave it to collect data every hour and add to the archive. This approach was used in all the data sets used in this paper.

\section{Setting up a TAGS sheet}
Setting up a TAGS sheet is relatively easy go to website (Hawksey, 2022), follow the Get TAGS link and choose which version you want to use. It is worth reading and following the guidance on the page about the app not verified. You will need an Twitter account to link to. 
Figure 1 shows the first page of the TAGS sheet after the hashtags you want to consider (box 2) have been eunning for a while.

So in summary the steps are
•	Go to https://tags.hawksey.info/ create a new spreadsheet via Get TAGS and select TAGS v6.1
•	Make a copy
•	In box 2 enter the search term e.g. #socmedhe21
•	Use the TAGS pull down menu select run now! 
•	Whole load of authorization
•	In TAGS menu update hourly
•	Share button: so all can view it. \url{https://docs.google.com/spreadsheets/d/1o49hgPpud99FdVD-DxShD1B_v4fuPdTZoTdrVZFCjXI/edit?usp=sharing} 
