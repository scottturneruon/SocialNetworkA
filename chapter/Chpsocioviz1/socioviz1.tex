\chapter{From socioviz to Gephi}
Starting with a Socioviz report instead of (or after viewing the report) viewing the report you can download it. A zip file of CSVs and gexf files is produced. \paragraph{}

\section{Dealing with gexf files}Going to concentrate on the gexf ones\paragraph{}

An example is shown in figure 1.1
\begin{figure}
    \centering
    \includegraphics[width=10cm]{chapter/Chpsocioviz1/gephi1.png}
    \caption{Selecting a gexf file}
    \label{fig:GexfSelection}
\end{figure}
\newline
Gexf files can be load straight into GEPHI without any real difficulty. Put when it is loaded it may not be in the format you want (see figure 1.2). \paragraph{}
\begin{figure}
    \centering 
    \includegraphics[width=10cm]{chapter/Chpsocioviz1/gephi2.png}
    \caption{After loading}
    \label{fig:afterload}
\end{figure}
It is difficult to see what is going on all the nodes are on top of each other. So now we play with the layout to get them spread out (see figure 1.3).\paragraph{}
\begin{figure}
    \centering
    \includegraphics[width=10cm]{chapter/Chpsocioviz1/gephi3.png}
    \caption{Change the Layout}
    \label{fig:layoutchange}
\end{figure}
Ok, but what do the points mean though. If you click on the T icon near the bottom of the screen it adds the names of the nodes (see figure 1.4).\paragraph{}
\begin{figure}
    \centering
    \includegraphics[width=10cm]{chapter/Chpsocioviz1/gephi5.png}
    \caption{Add the labels}
    \label{fig:addlabels}
\end{figure}
It is difficult to see so of the nodes so one trick is to make them all the same size using the Appear, node and Unique here they are all set to 10 (see figure 1.5).\paragraph{}
\begin{figure}
    \centering
    \includegraphics[width=10cm]{chapter/Chpsocioviz1/gephi6.png}
    \caption{Resize the nodes}
    \label{fig:nodes resize}
\end{figure}
